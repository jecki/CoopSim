
\documentclass[12pt,a4paper,USenglish]{article}
\usepackage{ae}
\usepackage{babel}
\usepackage[latin1]{inputenc}
\usepackage[T1]{fontenc}
\usepackage{t1enc}
\usepackage{type1cm}
\usepackage{graphicx}
\IfFileExists{url.sty}{\usepackage{url}}
                      {\newcommand{\url}{\texttt}}

%\makeatletter
%\makeatother

\sloppy

\begin{document}

\title{PyPlotter\\A Python/Jython Graph Plotting Package\\Manual}
\author{Eckhart Arnold}
\date{September, 6th 2015}

\maketitle

\tableofcontents{}


\section{Introduction}

{\sf PyPlotter} is a 2D graph plotting package for Python and Jython
(the java version of Python). It contains classes for drawing graphs
on a cartesian coordinate plain (with linar or logarithmic scale) and
for plotting 2D simplex diagrams. {\sf PyPlotter} supports different
GUI libraries and can easily adapted to other GUIs or output devices
by implementing a very simple driver interface. Currently (Version
0.9.2), tk, gtk, qt, wxWidgets, java awt and postscript are supported as
output devices.

Since Version 0.9.2 PyPlotter is Python 3.0 compatible. However, until 
the GUI toolkits pygtk and wxWidgets are available for Python 3.0, it is
only possible to use PyPlotter and Python 3.0 in connection with the tk 
toolkit.

\section{License }

The MIT License (MIT)

Copyright (c) 2004 Eckhart Arnold (eckhart\_arnold@hotmail.com, www.eckhartarnold.de)

Permission is hereby granted, free of charge, to any person obtaining a copy
of this software and associated documentation files (the "Software"), to deal
in the Software without restriction, including without limitation the rights
to use, copy, modify, merge, publish, distribute, sublicense, and/or sell
copies of the Software, and to permit persons to whom the Software is
furnished to do so, subject to the following conditions:

The above copyright notice and this permission notice shall be included in
all copies or substantial portions of the Software.

THE SOFTWARE IS PROVIDED "AS IS", WITHOUT WARRANTY OF ANY KIND, EXPRESS OR
IMPLIED, INCLUDING BUT NOT LIMITED TO THE WARRANTIES OF MERCHANTABILITY,
FITNESS FOR A PARTICULAR PURPOSE AND NONINFRINGEMENT. IN NO EVENT SHALL THE
AUTHORS OR COPYRIGHT HOLDERS BE LIABLE FOR ANY CLAIM, DAMAGES OR OTHER
LIABILITY, WHETHER IN AN ACTION OF CONTRACT, TORT OR OTHERWISE, ARISING FROM,
OUT OF OR IN CONNECTION WITH THE SOFTWARE OR THE USE OR OTHER DEALINGS IN
THE SOFTWARE.

\section{Quick Tutorial}

While the classes Graph.Cartesian and Simplex.Diagram are quite
versatile, it was a major aim of their development to make the usage
for beginners as simple as possible. To show you how to use these
classes, this tutorial contains two commented example programs.

\subsection{Example 1: Plotting a Graph}

In order to see the results of this example, either run the file
``Example1.py'' from the {\sf PyPlotter} directory or enter the following
lines at the python command prompt.

\begin{verbatim}
1:  import math
2:  from PyPlotter import tkGfx as GfxDriver # 'awtGfx' for jython
3:  from PyPlotter import Graph, Gfx
4:    
5:  gfx = GfxDriver.Window(title="Function Plotter")   
6:  gr = Graph.Cartesian(gfx, -4.0, -2.0, 4.0, 2.0)    
7:  gr.addPen("sin(x)", Gfx.RED_PEN)
8:  for x in gr.xaxisSteps(-4.0, 4.0):
9:       gr.addValue("sin(x)", x, math.sin(x))
10: gfx.waitUntilClosed()
\end{verbatim}

Thats all! If everything went right you should have seen a nice sine
curve on your display. Here is an explanation of what the program
does. Line 5 opens a window for graphical output. Then a new cartesian
graph is being created in this window. In line 7 a new pen is added to
the graph. Before you can draw anything onto the graph, you have to
add one or more pens. Every pen is identified by its unique name. By
default the graph as a caption where all pens are listed by their
names. To actually draw something on the graph, you have to add one or
more coordinate pairs to the graph with a given pen. The coordinate
pairs of a pen will then be connected with a continuous line in the
order they where added to the graph. This is done in line 8 and 9.  In
line 8 the method {\tt xaxisSteps} is called, which returns a list of
x values for a given range, each of which corresponds to exactly one
pixel on the screen.

Since Version 0.8.7 of PyPlotter the same can be done even simpler:

\begin{verbatim}
1:  import math
2:  from PyPlotter import Graph 
3:  gr = Graph.Cartesian(Graph.AUTO_GFX, -4.0, -2.0, 4.0, 2.0)    
4:  for x in gr.xaxisSteps(-4.0, 4.0):
5:       gr.addValue("sin(x)", x, math.sin(x))
6: gr.gfx.waitUntilClosed()
\end{verbatim}


\subsection{Example 2: Plotting a simplex diagram}

Here is a short code snipplet to demonstrate the use of a simplex diagram.
For the sake of brevity, the actual population dynamical function is
not contained. See file ``Example2.py'' for the full program.

\begin{verbatim}
1:  from PyPlotter import tkGfx as GfxDriver
2:  from PyPlotter import Simplex
3:
4:  gfx = GfxDriver.Window(title="Demand Game")   
5:  dynamicsFunction = lambda p: PopulationDynamics(p,DemandGame, 
6:                                                  e=0.0,noise=0.0)
7:  diagram = Simplex.Diagram(gfx, dynamicsFunction, "Demand Game",
8:                        "Demand 1/3", "Demand 2/3", "Demand 1/2")
9:  diagram.show()
10: gfx.waitUntilClosed()
\end{verbatim}

In order to draw a simplex diagram, you need to instantiate class
Simplex.Diagram (line 7) with a suitable population dynamical
function. Class {\tt Simplex.Diagram} is specifically designed for
visualizing population dynamics. If you want to use simplex diagrams
for another purpose, you should use the lower level class
Simplex.Plotter instead. The simplex diagram will not be drawn, unless
the {\tt show} method of class {\tt Simplex.Diagram} is called, as it
is done in line 9 of this example.

\section{Reference}

This reference of the {\sf PyPlotter} package does only cover the most high
level classes and functions of {\sf PyPlotter}. For a description of the
lower level classes and functions, see the doc strings is the source
code.

\subsection{Overview}

{\sf PyPlotter} basically consists of two parts, a front end part and
a back end part. The front end part comprises the high level classes
to plot cartesian graphs or simplex diagrams. These are the {\tt class
Cartesian} from the {\tt Graph} module and {\tt class Diagram} from
the {\tt Simplex} module. The backend part is a simple driver
interface that is defined in the {\tt Gfx} module. There exist several
implementations of this driver interface for different graphical user
interfaces. They are located in the modules named {\tt **Gfx}.

Package PyPlotter consists of the following Modules: 

\begin{description}

\item[Compatibility] A helper module to ensure compatibility
with different Python versions (Verions 2.1 through to Version 2.4) as
well as compatibility with Jython 2.1 .  

\item[Colors] A helper module for dealing with colors. If contains a
list of well distinguishable colors (useful if drawing many graphs on
one single plain) and a few filter functions that help assigning
similar color shades to graphs that belong to the same of several
groups.  

\item[Gfx] This module defines the driver interface
({\tt class Driver}). It also contains {\tt class Pen} to store a set 
of graphical attributes such as color, line width etc.

\item[**Gfx] These modules contain implementations of 
{\tt Gfx.Driver} for different GUIs. There are drivers for the following GUI
toolkits: 
  \begin{itemize}
    \item {\tt tkGfx} for the {\em tkinter} GUI toolkit that
          comes with the Python standard distribution.
    \item {\tt qtGfx} for the {\em qt} GUI toolkit
          (\url{www.riverbankcomputing.co.uk/software/pyqt/}).
          {\tt qtGfx} tries to import qt version 4, but falls
          back on version 3, if version 4 of qt is not present. 
    \item {\tt wxGfx} for the {\em wxWidgets} GUI
          toolkit (\url{www.wxwidgets.org}). 
    \item {\tt gtkGfx} for the {\em gtk} GUI toolkit (\url{www.pygtk.org}).
    \item {\tt awtGfx} for the Java {\em awt/swing} GUI toolkit under Jython,
          the Python version running under the Java JVM.
    \item {tt psGfx} for {\em postscript} output that can be written
          to a file.
  \end{itemize}

\item[Graph] Contains the high level class {\tt Cartesian} for drawing
  graphs on a cartesian plain. It also contains a number of
  intermediate level classes for mapping virtual to screen coordinates
  etc.

\item[Simplex] Contains the high level class {\tt Diagram} for drawing
  simplex diagrams of population dynamics. Within in this module also
  some intermediate classes for simplex drawing and coordinate
  transformation are implemented.

\end{description}

\subsection{Class {\tt Graph.Cartesian}} 

Class {\tt Graph.Cartesian} is versatile high level class for drawing
graphs on a cartesian plain. It supports linear and logarithmic scales
and automatic adjustment of the coordinate range as well as automatic
captioning.

\begin{description}

\item[\_\_init\_\_](self, gfx, x1, y1, x2, y2,\\ 
                    title = ``Graph'',  xaxis=''X'', yaxis=''Y'',\\
                    styleFlags = DEFAULT\_STYLE,\\
                    axisPen = Gfx.BLACK\_PEN, labelPen = Gfx.BLACK\_PEN,\\
                    titlePen = Gfx.BLACK\_PEN, captionPen = Gfx.BLACK\_PEN,\\
                    backgroundPen = Gfx.WHITE\_PEN,\\
                    region = REGION\_FULLSCREEN)\\ 
\\Initializes the class with the following parameters:

\begin{description}

\item[gfx] Gfx.Driver: The Gfx drivers used for drawing the graph. Use
AUTO\_PEN if you want the Graph.Cartesian object to find a suitable driver
(depending on the installed widget toolkits) on its own.
\item[x1,y1,x2,y2] floats: Coordinate range.
\item[title] string: Title string.
\item[xaxis, yaxis] strings: Axis descriptions.
\item[styleFlags] integer: Interpreted as a bitfield of flags that
       define the style of the graph. The following flags can be set:

\begin{description}
\item[AXISES, AXIS\_DIVISION, FULL\_GRID] Draw axises, axis divisions 
	and (or) a full grid.
\item[LABELS, CAPTION, TITLE] Draw axis labels, a caption with descriptions 
	(generated from the pen  names) below the graph, a title above 
	the graph.
\item[SHUFFLE\_DRAW, EVADE\_DRAW] Two different algorithms to
	allow for the visibility of overlapping graphs.
\item[LOG\_X, LOG\_Y] Use a logarithmic scale for the x or y axis respectively.
\item[KEEP\_ASPECT] Keep the aspect ratio of the coordinates.
\item[AUTO\_ADJUST] Automatically adjust the range of the graph
        when a point is added that falls outside the current range.
\end{description}
                    
\item[axisPen, labelPen, titlePen, captionPen, backgroundPen] Gfx.Pen: 
	Pens (sets of graphical attributes) for the respective elements 
        of the graph.
\item[region] 4-tuple of floats. The part of the screen that is used 
	for the graph. Example: (0.05, 0.05, 0.95, 0.95) would leave a 
	border of 5 \% of the screen size on each side.

\end{description}

\item[adjustRange](self, x1, y1, x2, y2) - Adjusts the range of the
  coordinate plane.

\item[setStyle](self, styleFlags=None, axisPen=None,\\ 
      labelPen=None, titlePen=None, captionPen=None,\\
      backgroundPen = None) - Changes the style of the graph. 
      Only parameters that are not {\tt None} will be changed.

\item[setTitle](self, title) - Changes the title of the graph.

\item[setLabels](self, xaxis=None, yaxis=None) - Changes the labels of
  the graph.

\item[resizedGfx](self) - Takes notice of a resized window.

\item[changeGfx](self, gfx) - Switch to another device context. This
  can be useful if you want to draw the current graph into a buffered
  image that you want to save on a disk. In this case you have to
  create the buffered image, create the Gfx driver for your buffered
  image, call changeGfx and then redraw. After that you can call
  changeGfx to switch back to the former output device.

\item[redrawGraph](self) - Redraws the graph, but not the caption,
  title or labels.

\item[redrawCaption](self) - Redraw only the caption of the graph.

\item[redraw](self) - Redraws the whole graph including, title, labels
  and the caption.

\item[reset](self, x1, y1, x2, y2) - Restarts with a new empty graph
  of the given range. All pens are removed.

\item[addPen](self, name, pen=AUTO\_GENERATE\_PEN,\\
  updateCaption=True) - Adds a new pen with name ``name'' and
  attributes ``pen'' to the graph.

\item[removePen](self, name, redraw=True) - Removes a pen from
        the graph. All coordinate pairs associated with this pen will
        be discarded.

\item[addValue](self, name, x, y) - Add the point (x,y) to the graph drawn with
	pen ``name''.

\item[peek](self, x, y) - Returns the graph coordinates of the screen 
	coordinates (x,y)

\item[xaxisSteps](self, x1, x2) - Returns a list of virtual x-coordinates 
	in the range [x1,x2] with one point for each screen pixel. This is
	especially useful when working with large range logarithmic scales.

\item[yaxisSteps](self, y1, y2) - Returns a list of virtual x-coordinates 
	in the range [y1,y2] with one point for each screen pixel.  This is
	especially useful when working with large range logarithmic scales.

\end{description}

\subsection{Class {\tt Simplex.Diagram}}

Class {\tt Simplex.Diagram} is a class for drawing simplex diagrams of
population dynamics of populations of three species. For simplex
diagrams dedicated to other purposes it is recommended to use the
lower level class {\tt Simplex.Plotter} instead.

\begin{description}

\item[\_\_init\_\_](self, gfx, function, title=''Simplex Diagram'',\\
                 p1=''A'', p2=''B'', p3=''C'', styleFlags = VECTORS,\\
                 raster = RASTER\_DEFAULT, density = -1,\\
                 color1 = (0.,1.,0.), color2 = (1.,0.,0.),\\
                 color3 = (0.,0.,1.), colorFunc = scaleColor,\\
                 titlePen = Gfx.BLACK\_PEN, labelPen = Gfx.BLACK\_PEN,\\
                 simplexPen=Gfx.BLACK\_PEN, backgroundPen=Gfx.WHITE\_PEN,\\
                 section=Graph.REGION\_FULLSCREEN)\\
\\Initializes the class with the following parameters:

\begin{description}

\item[gfx] Gfx.Driver: The Gfx drivers used for drawing the simplex diagram.

\item[function] f(p)->p*, where p and p' are 3 tuples of floats
        that add up to 1.0: Population dynamics function to be displayed
	in the simplex diagram.

\item[title, p1, p2, p3] strings: Strings to mark the title and the
        three corners of the diagram with.

\item[styleFlags] integer, interpreted as a bitfield of flags:
            The style or rather flavour of the simplex diagram.
            Presently three flavours are possible: {\tt VECTORS} for drawing
            the diagram as a vector field with many little arrows;
            {\tt TRAJECTORIES} for drawing pseudo trajectories; 
	    {\tt PATCHES} for drawing a patched diagram, where each 
	    point in the diagram has a unique color in the beginning. 
            From generation to generation, however, colors are adjusted 
	    such that every point ("patch") takes the color of the point it
            has moved to. This exposes areas of attraction in the
            diagram.

\item[raster] list of points (3-tuples of floats that add up to 1.0):
	The point raster of the simplex diagram. Suitable point rasters of
	varying density can be produced with the functions 
	{\tt Simplex.GenRaster} and {\tt Simplex.RandomGrid}.

\item[density] integer > 2:  The density of the points of the simplex 
	diagram. This is mainly useful in combination with style {\tt PATCHES},
	because this style does not use a raster.

\item[color1, color2, color3] (r,g,b)-tuples, where r,g and b are
        floats in range of [0.0, 1.0]: The three color parameters have a
	different meaning depending on the diagram style used. 
	 For patch diagrams these
            are the edge colors of the three edges of the diagram. For
            trajectory diagrams color1 is the starting color and color2
            is the color towards which later steps of the trajectory are
            shaded. For vector fields the range between color1 and
            color2 is used to indicate the strength of the vector field. 

\item[colorFunc] f(ca, cb, strength) -> c, where ca and cb are colors
            and strength is a float from [0, infinity]: This function
            produces a color shade from 'ca', 'cb' and 'strength',
            usually somewhere on the line between 'ca' and 'cb'. The
            parameter {\tt colorFunc} is not used for patches diagrams.

\item[titlePen, labelPen, simplexPen, backgroundPen] Gfx.Pen: Pens
            for the respective parts of the simplex diagram.

\item[section] 4-tuple of floats from then range [0.0, 1.0]: the
            part of the screen to be used for the diagram. 

\end{description}

\item[setStyle](self, styleFlags=None, titlePen=None,\\
  labelPen=None, simplexPen=None, backgroundPen=None) - Changes the
  style of the simplex diagram. It is not necessary to assign a value
  to all arguments of the functions. Those arguments that no value is
  assigned to will leave the respective class attributes untouched.

\item[setFunction](self, function) - Changes the population dynamics
  function that is visualized by the diagram. The change will only be
  visible after the method {\tt show} has been called.

\item[setRaster](self, raster) - Changes the raster of sample points.
  The change will only be visible after the method {\tt show} has been
  called.

\item[setDensity](self, density) - Generates a raster of uniformly
  distributed sample points (population distributions) with the given
  density.  The change will only be visible after the method {\tt
    show} has been called.

\item[changeColors](self, color1 = (0.,1.,0.), color2 = (1.,0.,0.),\\
  color3 = (0.,0.,1.), colorFunc=scaleColor) - Changes the colors of
  diagram, including a color modifying function. Note: The semantics
  of these paramters may differ depending on the visualizer used.  The
  change will only be visible after the method {\tt show} has been
  called.

\item[show](self, steps=-1) - Shows the diagram calculating 'steps'
  generations for dynamic diagrams (style {\tt TRAJECTORIES} or {\tt
    PATCHES}).

\item[showFixedPoints](self, color) - Shows candidates(!) for fixed
  points (only if style is {\tt PATCHES}).
        
\item[redraw](self) - Redraws the diagram.

\item[resizedGfx](self) - Takes notice of a resized graphics context
  and redraws the diagram.

\end{description}

\section{Implementing a new device driver for PyPlotter} 

Adapting {\sf PyPlotter} to a new GUI enviroment or to a new output
device is very easy. You only have to implement a class for the driver itself
that is derived from {\tt Gfx.Driver} and, optionally, also 
another very simple
standardized window class to open an output window (or context) on your 
GUI or device. The latter class
must be derived from class {\tt Gfx.Window}.

The driver class must implement the following methods from its parent
class {\tt Gfx.Drivers}: {\tt \_\_init\_\_}, {\tt resizedGfx}, {\tt
getSize}, {\tt getResolution}, {\tt setColor}, {\tt setColor}, {\tt
setLineWidth}, {\tt setLinePattern}, {\tt setFillPattern}, {\tt
setFont}, {\tt getTextSize}, {\tt drawLine}, {\tt fillPoly}, {\tt
writeStr}. Overriding the other methods or adding further methods is
optional and may lead to increased performance.

The window class must implement all methods of class {\tt Gfx.Window},
that is: {\tt \_\_init\_\_}, {\tt refresh}, {\tt quit}, {\tt
  waitUntilClosed}.

The already implemented drivers in modules {\tt awtGfx}, {\tt wxGfx},
{\tt tkGfx}, {\tt gtkGfx}, {\tt qtGfx} and {\tt psGfx} may serve as examples for
implementing new drivers.

\end{document}
